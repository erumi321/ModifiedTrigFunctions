\documentclass[11pt]{article}

\usepackage[utf8]{inputenc}
\usepackage[T1]{fontenc}
\usepackage{geometry}
\usepackage{mathtools}
\usepackage{amsfonts}
\usepackage{amsmath}
\usepackage{amssymb}
\usepackage[makeroom]{cancel}
\allowdisplaybreaks
\DeclareMathOperator{\custd}{\mathrm{d}_n}
\DeclareMathOperator{\custdsq}{\mathrm{d}_n}
\DeclareMathOperator{\custh}{\mathrm{h}_n}
\DeclareMathOperator{\custi}{\mathrm{i}}
\DeclareMathOperator{\period}{\frac{2\pi}{n}}
\DeclareMathOperator{\hp}{\frac{\pi}{n}}
\DeclareMathOperator{\iperiod}{\custi\frac{2\pi}{n}}
\DeclarePairedDelimiter\abs{\lvert}{\rvert}%

\renewcommand*\contentsname{Table of Contents}
\title{Triginometric Coefficient for Regular N-Sided Polygons}
\author{Eli Ruminer}
\date{}

\begin{document}
\begin{titlepage}
\clearpage\maketitle
\thispagestyle{empty}
\end{titlepage}

\tableofcontents
\newpage

\section{Introduction}
This paper derives a function that generates a coefficient that transforms sine and cosine into calculating points for any n-sided regular polygon. This is done through using inverse trigonometric functions to create a periodic function that increases and decreases linearly over each period. This allows the creation of a coefficent which when multiplied by sine and cosine serves to transform them into an equivalent function for non-circle polygons.\\
 After deriving the equation we then prove that it represents a point on the perimeter of a unit n-gon by simplifying the point-slope formula of a constructing line of the unit n-gon and two points between the modified sine and cosine showing they are the same. \\
Through this, we can derive a general formula for any shape using the sign function and the newly proven coefficient function. This general function requires the use of an "input" angle, which is calculated using x and y for each point, but due to this can be rotated by simply adding or subtracting a constant from the angle calculation inside the coefficient function's input.


\section{Theroems}
%\textbf{Theorem 1.1} \(\arccos(\cos(x))\neq\cos(\arccos(x))\).\\
%This can be shown when \(a=1\mathrm{ and }x=-\frac{\pi}{6}\) which solves to \(\frac{\pi}{6}\neq-\frac{\pi}{6}\).\\
\textbf{Theorem 1.1} \(\arccos(\cos(ax))\) is periodic over \(\frac{2\pi}{a*n}\).\\
\(\cos(ax)\) is periodic over \(\frac{2\pi}{a*n}\) thus \(\arccos(\cos(ax))\) must also be.\\
\textbf{Theorem 1.2} For \(x\in\mathbb{R}\) \(\arccos(\cos(ax))\in[0,\pi]\).\\
\(\cos(x)\) has a domain over \(\mathbb{R}\) and a range of [-1,1], \(\arccos(x)\) has a domain over [-1,1] and a range of \([0,\pi]\). Due to the range of cos being the domain of arccos, all real numbers can be an input of x, and all numbers on the interval [-1,1] can be output.\\
\textbf{Theorem 1.3} \(\cos(n(a+\period))=\cos(na)\)\\
\(\cos(n(a+\period))\) expands to \(\cos(na+2\pi))\) which due to cosine's periodic nature evaluates to \(\cos(na)\)\\
\textbf{Theorem 2.1} \(\sin(\alpha)-\sin(\beta)=2\sin(\frac{\alpha+\beta}{2})\cos(\frac{\alpha-\beta}{2})\)\\
\textbf{Theorem 2.2} \(\cos(\alpha)-\cos(\beta)=-2\sin(\frac{\alpha+\beta}{2})\sin(\frac{\alpha-\beta}{2})\)


\pagebreak
\section{Trigonometric\ Equations}
Let \(n=\{x:x\in\mathbb{Z}^+\ \mathrm{and}\ x\geq3\}\), and
\begin{gather*}
\custd(\theta)=\cos(\hp)\sec(\frac{\pi - \arccos(\cos(n\theta))}{n})\\
\cos_n(\theta)=\custd(\theta)\cos(\theta)\\
\sin_n(\theta)=\custd(\theta)\sin(\theta)
\end{gather*}
This is derived through the original functions:
\begin{gather*}
\custh(\theta)=\sin(\frac{\pi}{2}-\hp+(\theta\mod\period))-\sin(\frac{\pi}{2}-\hp)\\
\custd(\theta)=\frac{\custh(\theta)}{\sin(\frac{\pi}{2}-\hp+(\theta\mod\period))}\\
\cos_n(\theta)=(1-\custd(\theta))\cos(\theta)\\
\sin_n(\theta)=(1-\custd(\theta))\sin(\theta)
\end{gather*}
These functions follow the premise that any n-sided regular polygon (shape \(\rm s_n\)) centered at (0,0) has a circumcircle where each vertex touches the circumference of the circumcircle. \\
Arc \(a\) can be constructed which is \(\rm \frac{1}{n^{th}}\) of the circumcircle, its diameter is the same length as any line which constructs \(\rm s_n\), and it's circumference equals to the circumference of the section of the circumcircle spanning from adjacent vertices on \(\rm s_n\). \\
Thus, the height of \(a\) at a given point on a circle subtracted from the radius of the circle will result in the distance from the midpoint of \(\rm s_n\) to the point on the perimeter of \(\rm s_n\) which intersects the ray of angle \(\theta\) drawn in standard position.\\
\(\custh(\theta)\) serves to find the height of \(a\) at a position by splitting the circumcircle into \(\rm n^{ths}\) (resulting in a period of \(\period\)), finding the height of a point on \(a\) through sine and changing from the "left" vertex to the "right" vertex over the period of \(\period\) (the length of \(a\)). The use of mod makes the function periodic. It then subtracts by the height of a vertex so that \(\custh(\theta)=0\) when \(\theta=0\)\\
\(\custd(\theta)\) divides the height (\(\custh(\theta)\)) by the angle between the assumed intersection point on \(\rm s_n\) and the intersection point on the circumcircle to get the actual distance.\\
These functions then simplify to:
\begin{gather*}
\custd(\theta)=\cos(\hp)\sec(\hp-(\theta\mod\period))\\
\cos_n(\theta)=\custd(\theta)cos(\theta)\\
\sin_n(\theta)=\custd(\theta)sin(\theta)
\end{gather*}
Which is not algebraic due to mod (and the resultant floor function behind it), yet it can be noted that due to the nature of the original \(\custh(\theta)\), \(\theta\mod\period\) can be substituted for an equivalent function which has an equal period of \(\period\), increases linearly over \([0,\hp]\) to \([0,\hp]\) and decreases linearly at the same rate from \([\hp,\period]\).\\
These requirements can be met by \(\frac{\arccos(\cos(n\theta))}{n}\), which due to \textbf{Theorem 1.1} and \textbf{Theorem 1.2} is periodic over \(\period\) has a domain of \(n\theta\in\mathbb{R},\text{and thus } \theta\in\mathbb{R}\) and a range of \([0,\frac{\pi}{n}]\), thus the above equations can be changed to:
\begin{gather*}
\custd(\theta)=\cos(\hp)\sec(\frac{\pi-\arccos(\cos(n\theta))}{n})\\
\cos_n(\theta)=\custd(\theta)cos(\theta)\\
\sin_n(\theta)=\custd(\theta)sin(\theta)
\end{gather*}

\subsection{Identities}
\textbf{Identity 1.1} Let \(i=\mathbb{Z}\), due to \textbf{Theorem 1.3}:
\begin{gather*}
\cos_n(\theta+\iperiod) = \custd(\theta)\cos(\theta+\iperiod)\\
\sin_n(\theta+\iperiod) = \custd(\theta)\sin(\theta+\iperiod)
\end{gather*} \\
\textbf{Identity 1.2} When \(0\leq\theta\leq\period\) then \(\custd(\theta)=\cos(\hp)\sec(\hp-\theta)\)\\
When \(\theta\) only spans one period, \(\frac{\arccos(\cos(n\theta))}{n}\) is not required to make the function periodic.\\
\textbf{Identity 1.3} Let \(\custi=\mathbb{Z}\) then:
\begin{gather*}
\cos_n(\iperiod) = (1)\cos(\iperiod)=\cos(\iperiod)\\
\sin_n(\iperiod) = (1)\sin(\iperiod)=\sin(\iperiod)
\end{gather*}
This is true because \(\arccos(\cos(n\theta))\) evaluates to 0 whenever \(\theta=\iperiod\), which means \(\sec(\hp-\frac{\arccos(\cos(n\theta))}{n})\) simplifies to \(\sec(\hp)\) so \(\custd(\iperiod)=\sec(\hp)\cos(\hp)=1\)


\section{Proof}
Let \(\custi,n=\mathbb{Z}\). Shape \(\rm s_n\) is made of n vertices and n line segments with the \(\rm i^{th}\) vertex (\(\rm v_i\)) at the point \((\cos(\iperiod), \sin(\iperiod))\). The \(\rm i^{th}\) line segment, \(\rm L_i\), spans between \(\rm v_i\) and \(\rm v_{i+1}\) and is represented by the equation
\begin{gather*}
\mathrm{L_i}=\frac{\sin((\custi+1)\period)-\sin(\iperiod)}{\cos((\custi+1)\period)-\cos(\iperiod)}x+\sin(\iperiod)-\frac{\sin((\custi+1)\period)-\sin(\iperiod)}{\cos((\custi+1)\period)-\cos(\iperiod)}\cos(\iperiod)
\end{gather*}
which due to \textbf{Theorem 2.1} and \textbf{Theorem 2.2} simplifies to
\begin{gather*}
\mathrm{L_i}=-\cot((2\custi+1)\hp)x+\sin(\iperiod)+\cot((2\custi+1)\hp)\cos(\iperiod)
\end{gather*}
Let \(0\leq a<b\leq\period\), \(\alpha=a+\iperiod\), and \(\beta=b+\iperiod\): a "modified" line can be drawn with the equation \(\sin_n(\theta)\) and \(\cos_n(\theta)\) which would be written as
\begin{gather*}
y_m=\frac{\sin_n(\beta)-\sin_n(\alpha)}{\cos_n(\beta)-\cos_n(\alpha)}x+\sin_n(a)-\frac{\sin_n(\beta)-\sin_n(\alpha)}{\cos_n(\beta)-\cos_n(\alpha)}\cos_n(a)\\
\end{gather*}
the slope of \(y_m\) can be simplified through these steps:
\begin{gather*}
\frac{\custd(\beta)\sin(\beta)-\custd(\alpha)\sin(\alpha)}{\custd(\beta)\cos(\beta)-\custd(\alpha)\cos(\alpha)}\\
\cancel{\frac{\cos(\hp)}{\cos(\hp)}}\frac{\sec(\hp-b)\sin(\beta)-\sec(\hp-a)\sin(\alpha)}{\sec(\hp-b)\cos(\beta)-\sec(\hp-a)\cos(\alpha)}*\frac{\cos(\hp-a)\cos(\hp-b)}{\cos(\hp-a)\cos(\hp-b)}\\
\frac{\sin(\beta)\cos(\hp-a)-\sin(\alpha)\cos(\hp-b)}{\cos(\beta)\cos(\hp-a)-\cos(\alpha)\cos(\hp-b)}\\
\frac{\sin(\beta)(\cos(\hp)\cos(a)+\sin(\hp)\sin(a))-\sin(\alpha)(\cos(\hp)\cos(b)+\sin(\hp)\sin(b))}{\cos(\beta)(\cos(\hp)\cos(a)+\sin(\hp)\sin(a))-\cos(\alpha)(\cos(\hp)\cos(b)+\sin(\hp)\sin(b))}*\frac{\sec(\hp)}{\sec(\hp)}\\
\frac{\sin(\beta)(\cos(a)+\tan(\hp)\sin(a))-\sin(\alpha)(\cos(b)+\tan(\hp)\sin(b))}{\cos(\beta)(\cos(a)+\tan(\hp)\sin(a))-\cos(\alpha)(\cos(b)+\tan(\hp)\sin(b))} \ \ \ \mathrm{(\textbf{3.1})}
\end{gather*}
Splitting the equation up the numerator simplifies like so:
\begin{gather*}
\sin(\beta)(\cos(a)+\tan(\hp)\sin(a))-\sin(\alpha)(\cos(b)+\tan(\hp)\sin(b))\\\\
(\sin(b)\cos(\iperiod)+\cos(b)\sin(\iperiod))(\cos(a)+\tan(\hp)\sin(a))-\dotso\\\dotso(\sin(a)\cos(\iperiod)+\cos(a)\sin(\iperiod))(\cos(b)+\tan(\hp)\sin(b))\\\\
\cos(a)\sin(b)\cos(\iperiod)+\cancel{\sin(a)\sin(b)\cos(\iperiod)\tan(\hp)}+\cancel{\cos(a)\cos(b)\sin(\iperiod)}+\dotso\\\dotso\sin(a)\cos(b)\sin(\iperiod)\tan(\hp)-\sin(a)\cos(b)\cos(\iperiod)-\cancel{\sin(a)\sin(b)\cos(\iperiod)\tan(\hp)}-\dotso\\\dotso\cancel{\cos(a)\cos(b)\sin(\iperiod)}-\cos(a)\sin(b)\sin(\iperiod)\tan(\hp)\\\\
\cos(\iperiod)(\cos(a)\sin(b)-\sin(a)\cos(b))+\sin(\iperiod)\tan(\hp)(\sin(a)\cos(b)-\cos(a)\sin(b))\\\\
\sin(\iperiod)\tan(\hp)(\sin(a)\cos(b)-\cos(a)\sin(b))-\cos(\iperiod)(\sin(a)\cos(b)-\cos(a)\sin(b))\\\\
(\sin(\iperiod)\tan(\hp)-\cos(\iperiod))(\sin(a)\cos(b)-\cos(a)\sin(b))\\\\
(\sin(\iperiod)\tan(\hp)-\cos(\iperiod))\sin(a-b)
\end{gather*}
and the denominator simplifies to
\begin{gather*}
\cos(\beta)(\cos(a)+\tan(\hp)\sin(a))-\cos(\alpha)(\cos(b)+\tan(\hp)\sin(b))\\\\
(\cos(b)\cos(\iperiod)-\sin(b)\sin(\iperiod))(\cos(a)+\tan(\hp)\sin(a))-\dotso\\\dotso(\cos(a)\cos(\iperiod)-\sin(a)\sin(\iperiod))(\cos(b)+\tan(\hp)\sin(b))\\\\
\cancel{\cos(a)\cos(b)\cos(\iperiod)}+\sin(a)\cos(b)\cos(\iperiod)\tan(\hp)-\cos(a)\sin(b)\sin(\iperiod)-\dotso\\\dotso\cancel{\sin(a)\sin(b)\sin(\iperiod)\tan(\hp)}-\cancel{\cos(a)\cos(b)\cos(\iperiod)}-\cos(a)\sin(b)\cos(\iperiod)\tan(\hp)+\dotso\\\dotso\sin(a)\cos(b)\sin(\iperiod)+\cancel{\sin(a)\sin(b)\sin(\iperiod)\tan(\hp)}\\\\
(\cos(\iperiod)\tan(\hp))(\sin(a)\cos(b)-\cos(a)\sin(b))+\sin(\iperiod)(\sin(a)\cos(b)-\cos(a)\sin(b))\\
(\cos(\iperiod)\tan(\hp)+\sin(\iperiod)(\sin(a)\cos(b)-\cos(a)\sin(b))\\
(\cos(\iperiod)\tan(\hp)+\sin(\iperiod)\sin(a-b)
\end{gather*}
\pagebreak\\
meaning that we can now simplify the original fraction (denoted \textbf{3.1}) to become:
\begin{gather*}
\frac{(\sin(\iperiod)\tan(\hp)-\cos(\iperiod))\sin(a-b)}{(\cos(\iperiod)\tan(\hp)+\sin(\iperiod))\sin(a-b)}\\
\frac{\sin(\iperiod)\tan(\hp)-\cos(\iperiod)}{\cos(\iperiod)\tan(\hp)+\sin(\iperiod)} * \frac{\cos(\hp)}{\cos(\hp)}\\
\frac{\sin(\iperiod)\sin(\hp)-\cos(\iperiod)\cos(\hp)}{\cos(\iperiod)\sin(\hp)+\sin(\iperiod)\cos(\hp)}\\
\frac{-(\cos(\iperiod)\cos(\hp)-\sin(\iperiod)\sin(\hp))}{\sin(\hp)\cos(\iperiod)+\cos(\hp)\sin(\iperiod)}\\
\frac{-\cos(\iperiod+\hp)}{\sin(\iperiod+\hp)}\\
-\cot(\iperiod+\hp)\\
-\cot((2\custi+1)\hp)
\end{gather*}
so the equation of the "modified" line becomes:
\begin{gather*}
y_m=-\cot((2\custi+1)\hp)x+\sin_n(a+\iperiod)+\cot((2\custi+1)\hp)\cos_n(a+\iperiod)
\end{gather*}
This leads to the conclusion that when \(0\leq a<b\leq\period\) the slopes of \(y_m\) and \(y\) are equal. \\
If \(a=0\), due to \textbf{Identity 1.3}, then for any applicable value of \(b,\ y_m\) becomes:
\begin{gather*}
y_m=-\cot((2\custi+1)\hp)x+\sin(\iperiod)+\cot((2\custi+1)\hp)\cos(\iperiod)
\end{gather*}
Meaning that \(y_m=\mathrm{L_i}\) when \(a=0\) and due to \textbf{Identity 1.3}: \(\sin_n(a+\iperiod)=\sin(\iperiod) \mathrm{\ and\ } \cos_n(a+\iperiod)=\cos(\iperiod)\) thus the point \(\rm p_a\), located at \((\cos_n(a+\iperiod), \sin_n(a+\iperiod))\), is equal to \(\rm v_i\).\\
Because \(y_{m,a=0} = L_i\), point \(\rm p_b\) (located at \((\cos_n(b+\iperiod),\sin_n(b+\iperiod))\)) is always on line segment \(\rm L_i\), thus when \(a\neq0\), \(\rm p_a\) must also always be on line segment \(\rm L_i\).\\
This means that for any value \(x\) where \(0\leq x \leq\period\), the point \(\rm p_x\), located at \((\cos_n(x+\iperiod),\sin_n(x+\iperiod))\), falls on line \(\rm L_i\) of shape \(\rm s_n\), and since \(\custi\in\mathbb{Z}\), any point \(\rm p_\theta\), located at \((\cos_n(\theta),\sin_n(\theta))\), will always be on the perimeter of shape \(\rm s_n\).



\section{Shape Equations}
From these formulas, the equation for any regular polygon can be derived. The equation is found through a mutation of the ellipse equation by \(\custd(\theta)\):
\begin{gather*}
\frac{(x-h)^2}{a^2}+\frac{(y-k)^2}{b^2}=\custd(\theta)^2*1
\end{gather*}
To derive theta, one would start with \(\theta=\arctan(\frac{(y-k)a}{(x-h)b})\), which would find the angle at the correct point offset from the center, but due to x being squared, the shape is mirrored over \(x=h\), which will work for even sided polygons as they are symmetric over \(x=h\), but not for odd sided polygons due to their anti-symmetry (this is also the case over the line \(y=k\) but all shapes, regardless of side parity, are mirrored over \(y=k\)). This can be fixed by noting that the angle is calculated correctly in Quadrants I and IV (relative to \((h,k)\) as the origin), and \(\pi\) radians off from the expected angle in Quadrants II and III. Thus when \(x<h\), \(\pi\) must be added to \(\theta\). Using the sign function and modifying it to equal 1 when \(x<h\) and 0 when \(x\geq h\) results in the formula:
\begin{gather*}
-\frac{1}{2}(\frac{x-h}{\abs{x-h}}-1)
\end{gather*}
which when multiplied by \(\pi\) and added to \(\theta\) becomes:
\begin{gather*}
\arctan(\frac{(y-k)a}{(x-h)b})-\frac{\pi}{2}(\frac{x-h}{\abs{x-h}}-1)
\end{gather*}
meaning that the standard equation is:
\begin{gather*}
\frac{(x-h)^2}{a^2}+\frac{(y-k)^2}{b^2}=\custdsq(\arctan(\frac{(y-k)a}{(x-h)b})-\frac{\pi}{2}(\frac{x-h}{\abs{x-h}}-1))^2
\end{gather*}
As \(\custd\) takes an angle, a constant value can be added to \(\theta\) to make any shape rotate, allowing for easy rotation calculations.
\end{document}